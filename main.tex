\documentclass[twoside=false,DIV=8]{scrbook}

\usepackage[ngerman]{babel}
\usepackage{graphicx}
\usepackage[acronym]{glossaries}
\usepackage{csquotes}
\usepackage{subcaption}
\usepackage{lipsum}

\usepackage[
  style=apa,
  % autocite=footnote,
  % url=false,
]{biblatex}

\makeglossaries

\addbibresource{bibliography.bib}

\begin{document}

\titlehead{Name der Institution \hfill Sommersemester 2024}
\subject{Template}
\title{Wissenschaftliche Arbeit}
\subtitle{Hinweise zur Anfertigung von Seminar- und Abschlussarbeiten}
\publishers{Betreuung/Fachbereich}
\author{Mein Name}
\date{30.09.2024}

\newacronym{ki}{KI}{Künstliche Intelligenz}

\setcounter{page}{1}
\pagenumbering{Roman}

\maketitle
\tableofcontents

\listoffigures
\addcontentsline{toc}{chapter}{\listfigurename}

\listoftables
\addcontentsline{toc}{chapter}{\listtablename}

\renewcommand{\acronymname}{Abkürzungsverzeichnis}
\printglossary[type=\acronymtype]
\addcontentsline{toc}{chapter}{\acronymname}

\clearpage

\setcounter{page}{1}
\pagenumbering{arabic}

\include{einleitung}
\chapter{Literaturübersicht}

\section{Theoretische Grundlagen}

\Gls{ki} beschreibt ein großes Forschungsgebiet. Maschinelles Lernen ist ein Teilbereich hiervon. Ein konkretes Verfahren sind Entscheidungsbäume \autocite{knuth_2021}.

\lipsum[1]

\begin{equation}
  \mathit{R}^2=\frac{SSR}{SST}=
  \frac{\sum\nolimits \left(\hat{y}_i-\overline{y}\right)^2}{\sum\nolimits\left(y_i-\overline{y}\right)^2}=1-\frac{SSE}{SST}=1-\frac{\sum\nolimits\left(y_i-\hat{y}_i\right)^2}{\sum\nolimits \left(y_i-\overline{y}\right)^2}
\end{equation}

\lipsum[2]

\section{Stand der Forschung}

\lipsum[2]

\section{Relevante Konzepte und Modelle}

\lipsum[3-5]


\include{methodik}
\include{ergebnisse}

\chapter{Diskussion}

\section{Interpretation der Ergebnisse}

\lipsum[1]

\section{Implikationen für die Praxis}

\lipsum[2]

\section{Limitationen der Studie}

\lipsum[3]

\chapter{Schluss}

\section{Fazit}

\lipsum[1]

\section{Ausblick}

\lipsum[2]

\printbibliography

\end{document}
