\documentclass[twoside=false]{scrbook}

\usepackage[ngerman]{babel}
\usepackage{graphicx}
\usepackage[acronym]{glossaries}
\usepackage{csquotes}
\usepackage{subcaption}
\usepackage{lipsum}

\usepackage[
  style=apa,
  % autocite=footnote,
  % url=false,
]{biblatex}

\makeglossaries

\addbibresource{bibliography.bib}

\begin{document}

\titlehead{Name der Institution \hfill Sommersemester 2024}
\subject{Template}
\title{Wissenschaftliche Arbeit}
\subtitle{Hinweise zur Anfertigung von Seminar- und Abschlussarbeiten}
\publishers{Betreuung/Fachbereich}
\author{Dr. Tobias Knuth}
\date{30.09.2024}


\newacronym{html}{HTML}{HyperText Markup Language}

\setcounter{page}{1}
\pagenumbering{Roman}

\maketitle
\tableofcontents

\addcontentsline{toc}{chapter}{\listfigurename}
\listoffigures

\addcontentsline{toc}{chapter}{\listtablename}
\listoftables

\renewcommand{\acronymname}{Abkürzungsverzeichnis}
\addcontentsline{toc}{chapter}{\acronymname}

\printglossary[type=\acronymtype]

\clearpage

\setcounter{page}{1}
\pagenumbering{arabic}

\chapter{Einleitung}

\section{Motivation}

Viele Informationen zum Umgang mit \LaTeX finden sich in kostenfreien Büchern im Internet. Ein Beispiel ist das Buch \citetitle{oetiker_2022} \autocite{oetiker_2022}.

\lipsum[1]

\section{Zielsetzung}

\lipsum[2]

\section{Forschungsfragen}

\lipsum[3]

\section{Aufbau der Arbeit}

\lipsum[4-8]
\chapter{Literaturübersicht}

\section{Theoretische Grundlagen}

Ein Verfahren des maschinellen Lernens sind Entscheidungsbäume \autocite{knuth_2021}.

\lipsum[1]

\begin{equation}
  \mathit{R}^2=\frac{SSR}{SST}=
  \frac{\sum\nolimits \left(\hat{y}_i-\overline{y}\right)^2}{\sum\nolimits\left(y_i-\overline{y}\right)^2}=1-\frac{SSE}{SST}=1-\frac{\sum\nolimits\left(y_i-\hat{y}_i\right)^2}{\sum\nolimits \left(y_i-\overline{y}\right)^2}
\end{equation}

\lipsum[2]

\section{Stand der Forschung}

\lipsum[2]

\section{Relevante Konzepte und Modelle}

\lipsum[3-5]


\chapter{Methodik}

\section{Forschungsdesign}

\lipsum[1]

\section{Datenerhebung}

\lipsum[2]

\section{Datenanalyse}

\lipsum[3]

\chapter{Ergebnisse}

\section{Auswertung der Erhebung}

\lipsum[1-3]

\begin{figure}[t]
  \begin{minipage}{\linewidth}
    \begin{minipage}{0.5\linewidth}
      \centering
      \includegraphics[scale=0.4,angle=0]{example-image-a}
      \subcaption{Platzhalter A}
    \end{minipage}% <- sonst wird hier ein Leerzeichen eingefügt
    \begin{minipage}{0.5\linewidth}
      \centering
      \includegraphics[scale=0.4,angle=0]{example-image-b}
      \subcaption{Platzhalter B}
    \end{minipage}
    \caption{Zwei Buchstaben in einer Abbildung.}
  \end{minipage}
\end{figure}

\section{Vergleich mit bestehenden Studien}

\lipsum[4-8]


\chapter{Diskussion}

\section{Interpretation der Ergebnisse}

\lipsum[1]

\section{Implikationen für die Praxis}

\lipsum[2]

\section{Limitationen der Studie}

\lipsum[3]
\chapter{Schluss}

\section{Fazit}

\lipsum[1]

\section{Ausblick}

\lipsum[2]


\printbibliography

\end{document}