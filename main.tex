\documentclass[twoside=false]{scrbook}

\usepackage[ngerman]{babel}
\usepackage[acronym]{glossaries}

\makeglossaries

\begin{document}

\titlehead{Name der Institution \hfill Sommersemester 2024}
\subject{Template}
\title{Wissenschaftliche Arbeit}
\subtitle{Hinweise zur Anfertigung von Seminar- und Abschlussarbeiten}
\publishers{Betreuung/Fachbereich}
\author{Dr. Tobias Knuth}
\date{30.09.2024}

\newacronym{html}{HTML}{HyperText Markup Language}

\setcounter{page}{1}
\pagenumbering{Roman}

\maketitle
\tableofcontents

\addcontentsline{toc}{chapter}{\listfigurename}
\listoffigures

\addcontentsline{toc}{chapter}{\listtablename}
\listoftables

\renewcommand{\acronymname}{Abkürzungsverzeichnis}
\addcontentsline{toc}{chapter}{\acronymname}

\printglossary[type=\acronymtype]

\clearpage

\setcounter{page}{1}
\pagenumbering{arabic}

\chapter{Inhalt}

\Gls{html} ist die Sprache des Internets.

\chapter{Format}

\chapter{Zitieren}

\end{document}