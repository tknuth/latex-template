\chapter{Literaturübersicht}

\section{Theoretische Grundlagen}

\Gls{ki} beschreibt ein großes Forschungsgebiet. Maschinelles Lernen ist ein Teilbereich hiervon. Ein konkretes Verfahren sind Entscheidungsbäume \autocite{knuth_2021}.

\lipsum[1]

\begin{equation}
  \mathit{R}^2=\frac{SSR}{SST}=
  \frac{\sum\nolimits \left(\hat{y}_i-\overline{y}\right)^2}{\sum\nolimits\left(y_i-\overline{y}\right)^2}=1-\frac{SSE}{SST}=1-\frac{\sum\nolimits\left(y_i-\hat{y}_i\right)^2}{\sum\nolimits \left(y_i-\overline{y}\right)^2}
\end{equation}

\lipsum[2]

\section{Stand der Forschung}

\lipsum[2]

\section{Relevante Konzepte und Modelle}

\lipsum[3-5]

